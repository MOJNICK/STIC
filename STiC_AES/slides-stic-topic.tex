\subsection{Agenda}
%%%%%%%%%%%%%%%%%%%%%%%%%%%%%%%%%%%%%%%%%%%%%%%%%%%%%%%%%%%%%%%%%%%%%%%%%%%%%%%%%%%%%%%%%%%%%%%%%%% 
\begin{frame}
	\frametitle{Quantum crypanalysis}
		\framesubtitle{Agenda}
	\vspace{-1cm}
	\hspace{-1.5cm}{	
	\begin{description}
	\item[1.]{Bra-ket notation}
	\item[2.]{Quantum gates}
	\item[3.]{Grover's Database Search}
	\item[4.]{Shore's factorization algorithm}
	\begin{description}
	\hspace{-2.0cm}{• Fast modular exponentiation}\\
	\hspace{-2.0cm}{• Quantum Fourier Transform}\\
	\end{description}
	\end{description}
	}
\end{frame}

%%%%%%%%%%%%%%%%%%%%%%%%%%%%%%%%%%%%%%%%%%%%%%%%%%%%%%%%%%%%%%%%%%%%%%%%%%%%%%%%%%%%%%%%%%%%%%%%%%%	

\subsection{AES Origins}
%%%%%%%%%%%%%%%%%%%%%%%%%%%%%%%%%%%%%%%%%%%%%%%%%%%%%%%%%%%%%%%%%%%%%%%%%%%%%%%%%%%%%%%%%%%%%%%%%%% 
\begin{frame}
	\frametitle{Bra-ket notation}
		\framesubtitle{Definition}
		\vspace{-1cm}
	{\normalsize
	\hspace{0.5cm}{Bra–ket notation: $\braket{x|y}$ is a standard notation for describing quantum states. It can also be used to denote abstract vectors, linear functionals and scalar product in mathematics.}\\
	\vspace{0.4cm}
	\hspace{0.0cm}{The left part: $\bra{x}$, called the bra, is a row vector.}\\
	\hspace{0.0cm}{The right part: $\ket{y}$, called the ket, is a column vector.}\\
	\hspace{0.5cm}{}\\

	}
\end{frame}
%%%%%%%%%%%%%%%%%%%%%%%%%%%%%%%%%%%%%%%%%%%%%%%%%%%%%%%%%%%%%%%%%%%%%%%%%%%%%%%%%%%%%%%%%%%%%%%%%%%

%%%%%%%%%%%%%%%%%%%%%%%%%%%%%%%%%%%%%%%%%%%%%%%%%%%%%%%%%%%%%%%%%%%%%%%%%%%%%%%%%%%%%%%%%%%%%%%%%%% 
\begin{frame}
	\frametitle{Qbit}
		\framesubtitle{Definition}
		\vspace{-1cm}
	{\normalsize
	A pure qubit state is a linear superposition of the basis states. This means that the qubit can be represented as a linear\\ combination of $|0\rangle$ and +$|1\rangle $:\\ 
    $|\psi \rangle =\alpha |0\rangle +\beta |1\rangle$\\
  
    When we measure this qubit in the standard basis, the probability of outcome $|0\rangle$  is $|\alpha |^{2}$ and the probability of outcome $1\rangle$  is $|\beta |^{2}$. Because the absolute squares of the amplitudes equate to probabilities, it follows that
     $\alpha$ and $\beta$ must be constrained by the equation
$$|\alpha |^{2}+|\beta |^{2}=1$$\\
	}
\end{frame}
%%%%%%%%%%%%%%%%%%%%%%%%%%%%%%%%%%%%%%%%%%%%%%%%%%%%%%%%%%%%%%%%%%%%%%%%%%%%%%%%%%%%%%%%%%%%%%%%%%%

%%%%%%%%%%%%%%%%%%%%%%%%%%%%%%%%%%%%%%%%%%%%%%%%%%%%%%%%%%%%%%%%%%%%%%%%%%%%%%%%%%%%%%%%%%%%%%%%%%% 
\begin{frame}
	\frametitle{Gates}
		\framesubtitle{Definition}
		\vspace{-1cm}
	{\normalsize
	\hspace{0.5cm}{In quantum computing and specifically the quantum circuit model of computation, a quantum gate (or quantum logic gate) is a basic quantum circuit operating on a small number of qubits.}\\
	\vspace{0.4cm}
	\hspace{0.5cm}{.}\\
	}
\end{frame}
%%%%%%%%%%%%%%%%%%%%%%%%%%%%%%%%%%%%%%%%%%%%%%%%%%%%%%%%%%%%%%%%%%%%%%%%%%%%%%%%%%%%%%%%%%%%%%%%%%%

%%%%%%%%%%%%%%%%%%%%%%%%%%%%%%%%%%%%%%%%%%%%%%%%%%%%%%%%%%%%%%%%%%%%%%%%%%%%%%%%%%%%%%%%%%%%%%%%%%%
\begin{frame}
	\frametitle{Gates}
		\framesubtitle{Example}
	\vspace{0.5cm}
		\begin{figure}
		\centering
			\includegraphics[scale=1.3]{gates}
			\label{fig:gates gates}
		\end{figure}

\end{frame}
%%%%%%%%%%%%%%%%%%%%%%%%%%%%%%%%%%%%%%%%%%%%%%%%%%%%%%%%%%%%%%%%%%%%%%%%%%%%%%%%%%%%%%%%%%%%%%%%%%


\subsection{Grover} 
%%%%%%%%%%%%%%%%%%%%%%%%%%%%%%%%%%%%%%%%%%%%%%%%%%%%%%%%%%%%%%%%%%%%%%%%%%%%%%%%%%%%%%%%%%%%%%%%%%%
\begin{frame}
	\frametitle{Grover's database search}
		\framesubtitle{}

		{\normalsize
		\hspace{0.5cm}{Grover's database search uses ability of quantum computing to pararell process of qubits. The algorithm allows us to find selected element in unsorted set with complexity $\sqrt{n}$}\\
		}
		

\end{frame}
%%%%%%%%%%%%%%%%%%%%%%%%%%%%%%%%%%%%%%%%%%%%%%%%%%%%%%%%%%%%%%%%%%%%%%%%%%%%%%%%%%%%%%%%%%%%%%%%%%%

%%%%%%%%%%%%%%%%%%%%%%%%%%%%%%%%%%%%%%%%%%%%%%%%%%%%%%%%%%%%%%%%%%%%%%%%%%%%%%%%%%%%%%%%%%%%%%%%%%%
\begin{frame}
	\frametitle{Grover's database search}
		\framesubtitle{Scheme}
	\vspace{0.5cm}
		\begin{figure}
		\centering
		\includegraphics[width=10cm]{grdiffusion}
		\label{fig:grdiffusion grdiffusion}
	\end{figure}
\end{frame}
%%%%%%%%%%%%%%%%%%%%%%%%%%%%%%%%%%%%%%%%%%%%%%%%%%%%%%%%%%%%%%%%%%%%%%%%%%%%%%%%%%%%%%%%%%%%%%%%%%

%%%%%%%%%%%%%%%%%%%%%%%%%%%%%%%%%%%%%%%%%%%%%%%%%%%%%%%%%%%%%%%%%%%%%%%%%%%%%%%%%%%%%%%%%%%%%%%%%%%
\begin{frame}
	\frametitle{Fast exponentiation}
		\framesubtitle{}

		{\normalsize
		\hspace{0.5cm}{We can calculate $A^{B} mod C$ quickly, using modular multiplication rules:
$$A ^{2} mod C = (A * A) mod C = ((A mod C) * (A mod C)) mod C$$}\\
		}
		

\end{frame}
%%%%%%%%%%%%%%%%%%%%%%%%%%%%%%%%%%%%%%%%%%%%%%%%%%%%%%%%%%%%%%%%%%%%%%%%%%%%%%%%%%%%%%%%%%%%%%%%%%%

%%%%%%%%%%%%%%%%%%%%%%%%%%%%%%%%%%%%%%%%%%%%%%%%%%%%%%%%%%%%%%%%%%%%%%%%%%%%%%%%%%%%%%%%%%%%%%%%%%%
\begin{frame}
	\frametitle{Advanced Encryption Standard}
		\framesubtitle{3.MixColumns}
		\vfill
	
	\begin{block}{}
    	{Each column is represented as four-bytes vector.}\\
    \end{block}
    \begin{block}{}
		{Each column of State is replaced by a new column which is formed by multiplying that column by a certain 			matrix of elements of the field.}\\	
	\end{block}	
	    \begin{block}{}
		{Together with ShiftRows, MixColumns provides \textit{diffusion} in the cipher.}\\	
	\end{block}
		\begin{alertblock}{}
		{MixColumns step is used in every cycle \textbf{except} the last one cycle.}\\
		\end{alertblock}
\end{frame}
%%%%%%%%%%%%%%%%%%%%%%%%%%%%%%%%%%%%%%%%%%%%%%%%%%%%%%%%%%%%%%%%%%%%%%%%%%%%%%%%%%%%%%%%%%%%%%%%%%%

%%%%%%%%%%%%%%%%%%%%%%%%%%%%%%%%%%%%%%%%%%%%%%%%%%%%%%%%%%%%%%%%%%%%%%%%%%%%%%%%%%%%%%%%%%%%%%%%%%%
\begin{frame}
	\frametitle{Advanced Encryption Standard}
		\framesubtitle{3.MixColumns }
		
	\begin{block}{}
    	{It is also possible to see this operation as polynomial multiplication where each column is represented with 				 	polynomial a(x):}\\
		\hspace{0.5cm}{$a(x) = c(x).a(x) mod x^{4}+1= ({03}x^3 + {01}x^2 + {01}x + {02})
.(a_3x^3 + a_2x^2 + a_1x^1 + a_0) mod x^4 + 1 $}\\
		
	\end{block}
	\vfill
	\begin{block}{}
	{ $$c(x)= \left[
        \begin{array}{cccc}
         02 & 03\\
         01 & 02
         \end{array}
      \right] $$}
	\end{block}
\end{frame}
%%%%%%%%%%%%%%%%%%%%%%%%%%%%%%%%%%%%%%%%%%%%%%%%%%%%%%%%%%%%%%%%%%%%%%%%%%%%%%%%%%%%%%%%%%%%%%%%%%%

%%%%%%%%%%%%%%%%%%%%%%%%%%%%%%%%%%%%%%%%%%%%%%%%%%%%%%%%%%%%%%%%%%%%%%%%%%%%%%%%%%%%%%%%%%%%%%%%%%%
\begin{frame}
	\frametitle{Advanced Encryption Standard}
		\framesubtitle{Key Schedule: Rcon Table}
		\begin{table}
			\begin{center}
				\setlength\arrayrulewidth{1pt}
				
				\begin{tabular}{|c|c|c|c|}
					\hline 
					\multicolumn{4}{|c|}{Rcon Constants}\\
					\hline
					\hline Round 	& Constant(Rcon)		 &Round  		& Constant(Rcon)		\\
					\hline $1$	 	& $01$ $00$ $00$ $00$    &$6$			&$20$ $00$ $00$ $00$  	\\
					\hline $2$	 	& $02$ $00$ $00$ $00$    &$7$			&$40$ $00$ $00$ $00$  	\\
					\hline $3$	 	& $04$ $00$ $00$ $00$    &$8$			&$80$ $00$ $00$ $00$  	\\
					\hline $4$	 	& $08$ $00$ $00$ $00$    &$9$			&1B $00$ $00$ $00$  	\\
					\hline $5$	 	& $10$ $00$ $00$ $00$    &$10$			&$36$ $00$ $00$ $00$  	\\
					\hline
				\end{tabular}
			\end{center}
		\end{table}

		\vfill
\end{frame}
%%%%%%%%%%%%%%%%%%%%%%%%%%%%%%%%%%%%%%%%%%%%%%%%%%%%%%%%%%%%%%%%%%%%%%%%%%%%%%%%%%%%%%%%%%%%%%%%%%%
 

\subsection{Time for questions}
%%%%%%%%%%%%%%%%%%%%%%%%%%%%%%%%%%%%%%%%%%%%%%%%%%%%%%%%%%%%%%%%%%%%%%%%%%%%%%%%%%%%%%%%%%%%%%%%%%%
\begin{frame}
	
	\begin{center}
		\Huge \textbf{Time for questions}
	\end{center}

\end{frame}
%%%%%%%%%%%%%%%%%%%%%%%%%%%%%%%%%%%%%%%%%%%%%%%%%%%%%%%%%%%%%%%%%%%%%%%%%%%%%%%%%%%%%%%%%%%%%%%%%%%
\subsection{Bibliography}
%%%%%%%%%%%%%%%%%%%%%%%%%%%%%%%%%%%%%%%%%%%%%%%%%%%%%%%%%%%%%%%%%%%%%%%%%%%%%%%%%%%%%%%%%%%%%%%%%%%
\begin{frame}
	\frametitle{Advanced Encryption Standard}
		\framesubtitle{Bibliography}
	{\normalsize 
	
	Bibliography:\\	
	\vspace{0,2cm}
{- Joan Daemen, Vincent Rijmen, "The Design of Rijndael: AES – The Advanced Encryption Standard", Springer, 2002.}\\
\vspace{0,2cm}
{- Joshua Holden, "The Mathematics of Cryptography", Princeton University Press, 2017}\\
\vspace{0,2cm}
{- Federal Information Processing Standards Publication 197 : the official AES standard, United States National Institute of Standards and Technology, 2001}\\
\vspace{0,2cm}
{- Wikipedia, Advanced Encryption Standard, https://en.wikipedia.org/wiki/Advanced$\_$Encryption$\_$Standard}\\

	}
\end{frame}
%%%%%%%%%%%%%%%%%%%%%%%%%%%%%%%%%%%%%%%%%%%%%%%%%%%%%%%%%%%%%%%%%%%%%%%%%%%%%%%%%%%%%%%%%%%%%%%%%%%

\subsection{End}
%%%%%%%%%%%%%%%%%%%%%%%%%%%%%%%%%%%%%%%%%%%%%%%%%%%%%%%%%%%%%%%%%%%%%%%%%%%%%%%%%%%%%%%%%%%%%%%%%%%
\begin{frame}
	
	\begin{center}
		\Huge \textbf{Thank you for attention!}
	\end{center}

\end{frame}
%%%%%%%%%%%%%%%%%%%%%%%%%%%%%%%%%%%%%%%%%%%%%%%%%%%%%%%%%%%%%%%%%%%%%%%%%%%%%%%%%%%%%%%%%%%%%%%%%%%