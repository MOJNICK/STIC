\RequirePackage{flashmovie}
% it is neccessay to use "\RequirePackage{flashmovie}" because beamer
% also uses "\pdfminorversion". see flashmovie.sty for an explanation.

%---------------------------------------------
% jw-player-example
%---------------------------------------------

\documentclass[10pt]{beamer}

\usepackage[english]{babel}
\usepackage{hyperref}

\usetheme{Warsaw} % Warsaw,Hannover,boxes
\usecolortheme{rose} % orchid,lily,dolphin,beetle,crane
\usefonttheme{professionalfonts} % professionalfonts,serif
\useinnertheme{rounded} % rounded
\useoutertheme{shadow} % shadow,sidebar,split

%---------------------------------------------

\setbeamersize{text margin left=0.3cm}   % small margins
\setbeamersize{text margin right=0.3cm}

%---------------------------------------------
% for writing a document is is convenient to switch movies off. to do this,
% use "blank=1" as an option. in order to use the videos, use "blank=0".
% 
% to make thinks easy, use "blank=\myblank" as an option and define
% \myblank as suitable.

\def\myblank{0}
%\def\myblank{1}

%\def\mymovie{saturn5}
\def\mymovie{flagmoon}   % if the noise of saturn5 annoyes to much

%---------------------------------------------

\title[flashmovie.sty]{flashmovie.sty}
\author[\href{mailto:thartmann15@googlemail.com}{Timo Hartmann}]{\href{mailto:thartmann15@googlemail.com}{Timo Hartmann}}

\date[2010]{}

%---------------------------------------------

\begin{document}

\frame{

\titlepage

\begin{block}{\alert{Warning}}
It is recommended to use the latest available version of the Adobe Reader
to view PDF files generated with flashmovie.sty.
% Otherwise your Adobe Reader may die a sudden painfull death...
\end{block}

} % end frame

%---------------------------------------------

\begin{frame}

\frametitle{Basics}

\begin{minipage}[t]{6.1cm}
\vspace{0cm}
\flashmovie[width=6cm,height=5cm,engine=jw-player,auto=1,blank=\myblank]{\mymovie.mp4}
\end{minipage}
\begin{minipage}[t]{5.7cm}
\vspace{0cm}
This package allows direct embedding of flash movies into PDF files. It is
designed for use with pdflatex.
\vspace{0.5cm}

Basically it uses the fact that the Adobe Reader 9 contains an embedded Adobe Flash 
player which can be invoked with the ``rich media annotation'' feature which is described 
in ``Adobe Supplement to the ISO 32000 BaseVersion: 1.7 ExtensionLevel: 3''.
\vspace{0.5cm}

This means that you can only use flashmovie.sty in combination
with Adobe Reader 9 and upwards.

\end{minipage}

\vspace{0.5cm}

P.S.: This sample video is courtesy of the NASA ( \href{http://heasarc.gsfc.nasa.gov/Videos/historical/\mymovie.avi}{\mymovie.avi} ).

\end{frame}

%---------------------------------------------

\begin{frame}[fragile]

\frametitle{\href{http://www.longtailvideo.com/players/jw-flv-player}{JW Player}}

The source code used for the video on the previous page is:

\begin{verbatim}
\flashmovie[width=6cm,height=5cm,
              engine=jw-player,auto=1]{movie.mp4}
\end{verbatim}

This means that the movie is rendered with the 
\href{http://www.longtailvideo.com/players/jw-flv-player}{``JW Player''}
from longtail video (\href{http://www.longtailvideo.com}{www.longtailvideo.com}).

\vspace{0.5cm}

This is an open source flash player released under a
\href{http://creativecommons.org/licenses/by-nc-sa/3.0/}{non-commercial license},
which means its free for non-commercial use.

\end{frame}

%---------------------------------------------

\begin{frame}[fragile]

\frametitle{Issues with the flv-player}

\begin{itemize}

\item Sadly the \href{http://flv-player.net}{flv-player} does not work 
reliably with the latex package beamer. If you try it, the acrobat reader 
often crashes while trying to change the page. I have no idea 
what is the cause of this problem.

\item The only reliable way to use beamer seems to be directly embedding the videos as
      flash animations or to use the JW player.

\end{itemize}

\end{frame}

%---------------------------------------------

\begin{frame}[fragile]

\frametitle{Options for the \href{http://www.longtailvideo.com/players/jw-flv-player}{JW Player}}

\begin{minipage}[t]{4.1cm}
\vspace{0cm}
\flashmovie[width=4cm,height=3cm,engine=jw-player,auto=0,controlbar=0,blank=\myblank]{\mymovie.mp4}
\end{minipage}
\begin{minipage}[t]{7cm}
\vspace{0cm}
In this example the video is not started before the user clicks on it. The controlbar is disabled, too.
\begin{verbatim}
\flashmovie[width=4cm,height=3cm,
  engine=jw-player,auto=0,
  controlbar=0]{movie.mp4}
\end{verbatim}
\end{minipage}

\begin{minipage}[t]{4.1cm}
\vspace{0cm}
\flashmovie[width=4cm,height=3cm,engine=jw-player,auto=0,image=saturn.jpg,blank=\myblank,loop=1]{\mymovie.mp4}
\end{minipage}
\begin{minipage}[t]{7cm}
\vspace{0cm}
Here additionally an image is displayed before the movie starts. The movie also loops.
\begin{verbatim}
\flashmovie[width=4cm,height=3cm,
  engine=jw-player,auto=0,
  image=saturn.jpg,loop=1]
  {movie.mp4}
\end{verbatim}
\end{minipage}

\end{frame}

%---------------------------------------------

\end{document}
