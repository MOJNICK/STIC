%%%%%%%%%%%%%%%%%%%%%%%%%%%%%%%%%%%%%%%%%%%%%%%%%%%%%%%%%%%%%%%%%%%%%%%%%%%%%%%%%%%%%%%%%%%%%%%%%%%
\subsection{AES Purpose} 
\begin{frame}
	\frametitle{Purpose of creating AES cipher}
		
\end{frame}
%%%%%%%%%%%%%%%%%%%%%%%%%%%%%%%%%%%%%%%%%%%%%%%%%%%%%%%%%%%%%%%%%%%%%%%%%%%%%%%%%%%%%%%%%%%%%%%%%%%

%%%%%%%%%%%%%%%%%%%%%%%%%%%%%%%%%%%%%%%%%%%%%%%%%%%%%%%%%%%%%%%%%%%%%%%%%%%%%%%%%%%%%%%%%%%%%%%%%%%
\subsection{Requirements for AES} 
\begin{frame}
	\frametitle{Requirements for AES}
	%\vspace{-2cm}
		In the selection process, NIST asked for: \\
		{\footnotesize 
		\hspace{0.5cm}{- A block cipher, block length: 128 bit}\\
		\hspace{0.5cm}{- Key length: 128, 192 and 256 bit}\\
		\hspace{0.5cm}{- Suitability for hardware and software}\\
		}
		\vspace{0.2cm}
		{\normalsize NIST platform used to test candidate cipher algorithms:}\\
		{\footnotesize
		\hspace{0.5cm}{- PC IBM-compatible,Pentium Pro 200 MHz, 64 MB RAM, WINDOWS 95}\\
		\hspace{0.5cm}{- Borland C++ 5.0 compiler and Java Development Kit (JDK) 1.1}\\
		}
		\vspace{0.2cm}
		{\normalsize NIST selection of the winning algorithm based on:}\\
		{\footnotesize
		\hspace{0.5cm}{- Security}\\
		\hspace{0.5cm}{- Efficient implementation \textbf{both} in hardware and software}\\
		\hspace{0.5cm}{- Code length and memory utilization}\\
		}	
\end{frame}
%%%%%%%%%%%%%%%%%%%%%%%%%%%%%%%%%%%%%%%%%%%%%%%%%%%%%%%%%%%%%%%%%%%%%%%%%%%%%%%%%%%%%%%%%%%%%%%%%%%

%%%%%%%%%%%%%%%%%%%%%%%%%%%%%%%%%%%%%%%%%%%%%%%%%%%%%%%%%%%%%%%%%%%%%%%%%%%%%%%%%%%%%%%%%%%%%%%%%%%
\subsection{AES selection process} 
\begin{frame}
	\frametitle{AES selection process}
		
\end{frame}
%%%%%%%%%%%%%%%%%%%%%%%%%%%%%%%%%%%%%%%%%%%%%%%%%%%%%%%%%%%%%%%%%%%%%%%%%%%%%%%%%%%%%%%%%%%%%%%%%%%
	
%%%%%%%%%%%%%%%%%%%%%%%%%%%%%%%%%%%%%%%%%%%%%%%%%%%%%%%%%%%%%%%%%%%%%%%%%%%%%%%%%%%%%%%%%%%%%%%%%%%
\begin{frame}
	\frametitle{AES selection process}
\end{frame}
%%%%%%%%%%%%%%%%%%%%%%%%%%%%%%%%%%%%%%%%%%%%%%%%%%%%%%%%%%%%%%%%%%%%%%%%%%%%%%%%%%%%%%%%%%%%%%%%%%%

\subsection{Subtopic 1} % Change, please!

%%%%%%%%%%%%%%%%%%%%%%%%%%%%%%%%%%%%%%%%%%%%%%%%%%%%%%%%%%%%%%%%%%%%%%%%%%%%%%%%%%%%%%%%%%%%%%%%%%%
\begin{frame}
	\frametitle{Title of the Slide}
		\framesubtitle{Slide without Bullets}

	Text:

	\vfill

	\begin{block}{}
		Blocks are good for important notions.
	\end{block}

	\vfill

	\begin{alertblock}{}
		Alertblocks are even better to catch the attention.
	\end{alertblock}

	\vfill

	\begin{alertblock}{Title}
		You can have the title in the (alert)block.
	\end{alertblock}

\end{frame}
%%%%%%%%%%%%%%%%%%%%%%%%%%%%%%%%%%%%%%%%%%%%%%%%%%%%%%%%%%%%%%%%%%%%%%%%%%%%%%%%%%%%%%%%%%%%%%%%%%%

\subsection{Subtopic 2} % Change, please!

%%%%%%%%%%%%%%%%%%%%%%%%%%%%%%%%%%%%%%%%%%%%%%%%%%%%%%%%%%%%%%%%%%%%%%%%%%%%%%%%%%%%%%%%%%%%%%%%%%%
\begin{frame}
	\frametitle{Title of the Slide}
		\framesubtitle{Slide with a Figure from a File}

	\begin{figure}
		\includegraphics[width=6cm]{TIiK_channel_mutual_new}
		\caption{Caption of the figure}
	\end{figure}

	\vspace{-0.5cm}

	\tiny I am sorry that the example figures and the table concern the information theory \smiley

\end{frame}
%%%%%%%%%%%%%%%%%%%%%%%%%%%%%%%%%%%%%%%%%%%%%%%%%%%%%%%%%%%%%%%%%%%%%%%%%%%%%%%%%%%%%%%%%%%%%%%%%%%

%%%%%%%%%%%%%%%%%%%%%%%%%%%%%%%%%%%%%%%%%%%%%%%%%%%%%%%%%%%%%%%%%%%%%%%%%%%%%%%%%%%%%%%%%%%%%%%%%%%
\begin{frame}
	\frametitle{Title of the Slide}
		\framesubtitle{Slide with a Figure Drawn with \texttt{tikz}}

	\vspace{-0.5cm}

	\begin{changemargin}{-1cm}{-1cm}
		\begin{figure}
			\begin{tikzpicture}[x=2.5cm,y=2cm]
				\node[draw,thick,rectangle,fill=blue!25,align=center,text width = 1.5cm] (SOURCE) at (0,0) {\bfseries Source}; 
				\node[draw,thick,rectangle,fill=green!25,label=above:{\it Compression},align=center,text width = 1.5cm] (SOURCE_ENCODER) at (1,0) {\bfseries Source encoder}; 
				\node[draw,thick,rectangle,fill=magenta!25,label=above:{\it Security},align=center,text width = 2.00cm] (ENCRYPTION) at (2,0) {\bfseries Encryption}; 
				\node[draw,thick,rectangle,fill=red!25,align=center,text width = 1.5cm] (CHANNEL_ENCODER) at (3,0) {\bfseries Channel encoder};
				\node[align=center,text width = 1.5cm,yshift=1.05cm] at (CHANNEL_ENCODER) {\it Error\\protection}; 
				\node[draw,thick,rectangle,fill=gray!25,align=center,text width = 2.25cm] (MODULATOR) at (4,0) {\bfseries Modulator}; 
				\node[draw,thick,rectangle,fill=yellow!25,align=center,text width = 1.5cm] (CHANNEL) at (4,-1) {\bfseries Channel}; 
				\node[draw,thick,rectangle,fill=gray!25,align=center,text width = 2.25cm] (DEMODULATOR) at (4,-2) {\bfseries Demodulator}; 
				\node[draw,thick,rectangle,fill=red!25,align=center,text width = 1.5cm] (CHANNEL_DECODER) at (3,-2) {\bfseries Channel decoder}; 
				\node[draw,thick,rectangle,fill=magenta!25,align=center,text width = 2.00cm] (DECRYPTION) at (2,-2) {\bfseries Decryption}; 
				\node[draw,thick,rectangle,fill=green!25,align=center,text width = 1.5cm] (SOURCE_DECODER) at (1,-2) {\bfseries Source decoder}; 
				\node[draw,thick,rectangle,fill=blue!25,align=center,text width = 1.5cm] (SINK) at (0,-2) {\bfseries Sink}; 
				\node[starburst, fill=yellow, draw=red,line width=1pt,shift={(0.25,0.60cm)}] (NOISE) at (CHANNEL) {\color{red} \bf \tiny !};
	% 			\node[align=center,text width = 1.5cm] at (NOISE) {\color{red} Noise,\\errors,\\frauds};
				\draw[thick,->] (SOURCE) -- (SOURCE_ENCODER) ;
				\draw[thick,->] (SOURCE_ENCODER) -- (ENCRYPTION);
				\draw[thick,->] (ENCRYPTION) -- (CHANNEL_ENCODER);
				\draw[thick,->] (CHANNEL_ENCODER) -- (MODULATOR);
				\draw[thick,->] (MODULATOR) -- (CHANNEL);
				\draw[thick,->] (CHANNEL) -- (DEMODULATOR);
				\draw[thick,->] (DEMODULATOR) -- (CHANNEL_DECODER);
				\draw[thick,->] (CHANNEL_DECODER) -- (DECRYPTION);
				\draw[thick,->] (DECRYPTION) -- (SOURCE_DECODER);
				\draw[thick,->] (SOURCE_DECODER) -- (SINK);
				\coordinate (c1) at ($(CHANNEL_DECODER.south) + (0,-0.25cm)$);
				\draw[thick] (CHANNEL_DECODER) -- (c1);
				\draw[thick,->] (c1) -| (DEMODULATOR.south);
				\node at (3.5,-2.5) {\it Iterative decoding};
				\node[align=center,text width = 1.5cm] (SCT) at (1,-1) {Source\\coding\\theorem};
				\draw[->] (SCT) -- +(0,1.25cm);
				\draw[->] (SCT) -- +(0,-1.25cm);
				\node[align=center,text width = 1.5cm] (CCT) at (3,-1) {Channel\\coding\\theorem};
				\draw[->] (CCT) -- +(0,1.25cm);
				\draw[->] (CCT) -- +(0,-1.25cm);
				\draw[->] (CCT) -- +(0.5,0cm);
			\end{tikzpicture}
		\end{figure}		
	\end{changemargin}

	\vspace{-0.35cm}

	\tiny Source: \bibentry{Moon05a}.

\end{frame}
%%%%%%%%%%%%%%%%%%%%%%%%%%%%%%%%%%%%%%%%%%%%%%%%%%%%%%%%%%%%%%%%%%%%%%%%%%%%%%%%%%%%%%%%%%%%%%%%%%%

%%%%%%%%%%%%%%%%%%%%%%%%%%%%%%%%%%%%%%%%%%%%%%%%%%%%%%%%%%%%%%%%%%%%%%%%%%%%%%%%%%%%%%%%%%%%%%%%%%%
\begin{frame}
	\frametitle{Title of the Slide}
		\framesubtitle{Slide with a Table}
	
		\begin{table}
			\caption{Caption of the Table}
			\begin{center}
				\setlength\arrayrulewidth{2pt}
					\begin{tabular}{|>{\columncolor[rgb]{0,0.9,0.3}\color{white}}c|>{\columncolor[rgb]{0,0.9,0.1}\color{white}}c|}
					\hline
					\rowcolor[rgb]{0,0.5,0} \bfseries Data  & \bfseries Entropy \\
					\hline%
					\hline Plain text                        	  & $4{.}347$\\
					\hline Native executables	                   & $5{.}099$\\
					\hline Packed executables                   & $6{.}801$\\
					\hline Encrypted executables                & $7{.}175$\\
					\hline
				\end{tabular}
			\end{center}
		\end{table}

		\vfill

		\tiny Source: \bibentry{Lyda07a}.
\end{frame}
%%%%%%%%%%%%%%%%%%%%%%%%%%%%%%%%%%%%%%%%%%%%%%%%%%%%%%%%%%%%%%%%%%%%%%%%%%%%%%%%%%%%%%%%%%%%%%%%%%%



%%%%%%%%%%%%%%%%%%%%%%%%%%%%%%%%%%%%%%%%%%%%%%%%%%%%%%%%%%%%%%%%%%%%%%%%%%%%%%%%%%%%%%%%%%%%%%%%%%%
\begin{frame}
	
	\begin{center}
		\Huge \textbf{Thank you for attention!}
	\end{center}

\end{frame}
%%%%%%%%%%%%%%%%%%%%%%%%%%%%%%%%%%%%%%%%%%%%%%%%%%%%%%%%%%%%%%%%%%%%%%%%%%%%%%%%%%%%%%%%%%%%%%%%%%%