%%%%%%%%%%%%%%%%%%%%%%%%%%%%%%%%%%%%%%%%%%%%%%%%%%%%%%%%%%%%%%%%%%%%%%%%%%%%%%%%%%%%%%%%%%%%%%%%%%%
\begin{frame}
	\frametitle{Group Theory}
		\framesubtitle{Abellian Group}

		{\normalsize
		\hspace{0.5cm}{In abstract algebra, an abelian group, also called a commutative group, is a group in which the result of applying the group operation to two group elements does not depend on the order in which they are written.}\\
		}
		

\end{frame}
%%%%%%%%%%%%%%%%%%%%%%%%%%%%%%%%%%%%%%%%%%%%%%%%%%%%%%%%%%%%%%%%%%%%%%%%%%%%%%%%%%%%%%%%%%%%%%%%%%%

%%%%%%%%%%%%%%%%%%%%%%%%%%%%%%%%%%%%%%%%%%%%%%%%%%%%%%%%%%%%%%%%%%%%%%%%%%%%%%%%%%%%%%%%%%%%%%%%%%%
\begin{frame}
	\frametitle{Group Theory}
		\framesubtitle{Multiplicative group of integers modulo n}

		{\normalsize
		\hspace{0.5cm}{Multiplicative group of integers modulo n is an abelian group. The set of classes relatively prime to n is closed under multiplication: $$gcd(a, n) = \neg 1 \quad \textrm{and} \quad gcd(b, n) = 1 \quad => \quad gcd(ab, n) = 1$$}\\
		}
		

\end{frame}
%%%%%%%%%%%%%%%%%%%%%%%%%%%%%%%%%%%%%%%%%%%%%%%%%%%%%%%%%%%%%%%%%%%%%%%%%%%%%%%%%%%%%%%%%%%%%%%%%%%